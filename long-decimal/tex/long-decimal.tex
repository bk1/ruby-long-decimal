% LaTeX
% $Header: /var/cvs/long-decimal/long-decimal/tex/long-decimal.tex,v 1.3 2011/02/09 22:51:52 bk1 Exp $
% $Name:  $
%
\documentclass[10pt,a4paper]{article}
\usepackage{isodate}
\usepackage[T1]{fontenc}
\usepackage[latin1]{inputenc}
\usepackage{graphics}
\usepackage{amsmath,amssymb}
\usepackage[a4paper,margin=3cm,footskip=.5cm]{geometry}

\title{The Ruby library {\slshape long-decimal}}
\author{Karl Brodowsky\\
IT Sky Consulting GmbH\\
Switzerland}

\date{2011-02-06}

%\makeatletter
%\makeatother

\setlength{\topmargin}{0mm}
%\setlength{\leftmargin}{10mm}
\setlength{\oddsidemargin}{-11mm}
\setlength{\evensidemargin}{-11mm}
\setlength{\parindent}{0mm}

%\newfont{\ocrb}{ocrb10}

\def\ld{\mathrm l}
\def\ldq{\mathrm q}

\begin{document}
\sffamily
\maketitle

\begin{abstract}

The goal of the Ruby-library {\slshape long-decimal}\/ is to provide a new
numerical type {\slshape LongDecimal\/}\/ and arithmetic operations to deal
with this type.

\end{abstract}

\section{Introduction}

Ruby supports non-integral numbers with the built-in types {\slshape
  Float\/} and {\slshape Rational\/} and {\slshape BigDecimal\/}.
While these are very useful for many purposes, the development of
finance application requires the availability of a type like
{\slshape LongDecimal\/}, that allows explicit control of the rounding and uses a
decimal representation for the fractional part.  An instance of
{\slshape LongDecimal\/} can be represented as a pair of integral
numbers $(n, d)$, where the value of the number is $\frac{n}{10^d}$.
The ring-operations $+$, $-$ and $*$ can be trivially defined in a way
that does not loose any information (and thus does not require any
rounding).

Division requires some additional thought, because the exact quotient
of two {\slshape LongDecimal\/}s can very well be expressed as
Rational, but not always as {\slshape LongDecimal\/}.  A supplementary
numeric type {\slshape LongDecimalQuot\/} has been introduced to
support storing pairs $(r, d)$, where $r$ is a rational number and $d$
is an estimation about the significant digits after the decimal point.
The represented value is the rational number $r$.

Calculation of roots and trancendental functions require additional
information as input parameters to control the number of digits after
the decimal point and the rounding rules for achieving the result.

In the remainder of this document we will write $\ld(n, d)$ for the
{\slshape LongDecimal\/}-number represented by the pair $(n, d)$ in
the manner described above and $\ldq(r, d)$ for the {\slshape
  LongDecimalQuot\/}-number represented by the pair $(r, n)$.  It
needs to be observed that the part $d$ in both cases carries
information about the precision of the number in terms of significant
digits after the decimal point.

\pagebreak

\section{Ring Operations}

The ring operations $+$, $-$ and $*$ and obvious derived operations
like negation are defined like this:

$$\bigwedge_{e,d \in \mathbb{N}_0} \bigwedge_{m, n \in \mathbb{Z}}
\ld(m,d) + \ld(n,e) = \ld(m \cdot 10^{\max(d, e)-d} + n \cdot 10^{\max(d, e)-e}, \max(d, e))$$
$$\bigwedge_{e,d \in \mathbb{N}_0} \bigwedge_{m, n \in \mathbb{Z}}
\ld(m,d) - \ld(n,e) = \ld(m \cdot 10^{\max(d, e)-d} - n \cdot 10^{\max(d, e)-e}, \max(d, e))$$
$$\bigwedge_{e,d \in \mathbb{N}_0} \bigwedge_{m, n \in \mathbb{Z}}
\ld(m,d) \cdot \ld(n,e) = \ld(m n, d+e)$$

\pagebreak

\section{Rounding Operations}

The library supports two rounding operations that can be applied both
to {\slshape LongDecimal\/} and {\slshape LongDecimalQuot\/}.  In the
latter case it implies a conversion to {\slshape LongDecimal\/}.  The
normal rounding function {\slshape round\_to\_scale\/} changes the
precision to the given value, while keeping the value expressed by the
number approximately the same.

The optional second parameter describes how rounding will be done, if the
process looses information.  It defaults to {\slshape
ROUND\_UNNECESSARY\/}, in which case an exception is raised whenever
a value changing rounding process would be required.

The other rounding operations can be grouped into two groups:
ROUND\_UP, ROUND\_DOWN, ROUND\_CEILING and ROUND\_FLOOR already answer
the question of how to round completely because they use the values
that result from rounding as boundaries and obviously round these to
themselves.

The other rounding operations put a boundary somewhere in the middle
between allowed rounded values, but they leave the question of how to
round the boundary itself to be answered as well.  This is expressed
by having a major rounding mode that defines where the boundaries lie
and a minor rounding mode that is applicable for values that lie
exactly on the boundary. In some cases this cannot happen, because
boundaries are irrational and {\slshape LongDecimal\/} (and {\slshape LongDecimalQuot\/}) can
only express (some) rational numbers, but for the sake of completeness
and applicability to other representations, that might allow some
irrational values, this is done in a uniform way.  Since rounding
operations can also apply to internal intermediate results, it is a
good idea not to constrain their definition to rational numbers.

Major rounding modes are

\begin{tabular}{|l|p{100mm}|}
\hline
  MAJOR\_UP           & always round in such a way that the absolute value does not decrease.No minor rounding mode needed.\\
\hline
  MAJOR\_DOWN         & always round in such a way that the absolute value does not increase. No minor rounding mode needed.\\
\hline
  MAJOR\_CEILING      & always round in such a way that the rational value does not decrease. No minor rounding mode needed.\\
\hline
  MAJOR\_FLOOR        & always round in such a way that the rational value does not increase. No minor rounding mode needed.\\
\hline
  MAJOR\_HALF         & round to the nearest available value.  The boundary is the arithmetic mean of the two adjacent available values.\\
\hline
  MAJOR\_GEOMETRIC    & round to one of the two adjacent available values $x,y$ and use their geometric mean $\sqrt{xy}$ as boundary.  For negative $x,y$ use the negated square root instead.\\
\hline
  MAJOR\_HARMONIC     & round to one of the two adjacent available values $x,y$ and use their harmonic mean $\frac{2xy}{x+y}$ as boundary.\\
\hline
  MAJOR\_QUADRATIC    & round to one of the two adjacent available values $x,y$ and use their quadratic mean $\sqrt{\frac{x^2+y^2}{2}}$ as boundary.  If $x,y\le 0$ use the negated square root instead.\\
\hline
  MAJOR\_CUBIC        & round to one of the two adjacent available values $x,y$ and use their cubic mean $\sqrt[3]{\frac{x^3+y^3}{2}}$ as boundary.\\
\hline
  MAJOR\_UNNECESSARY  & raise an exception if value changing rounding would be needed\\
\hline
\end{tabular}

\pagebreak

Minor rounding modes are


\begin{tabular}{|l|p{100mm}|}
\hline
{\bfseries rounding mode}&{\bfseries description}\\
\hline
  MINOR\_UNUSED  & no minor rounding mode applies, can only be combined with ROUND\_UP, ROUND\_DOWN, ROUND\_CEILING and ROUND\_FLOOR.\\
\hline
  MINOR\_UP      & round values exactly on the boundary by increasing the absolute value (away from zero).\\
\hline
  MINOR\_DOWN    & round values exactly on the boundary by decreasing the absolute value (towards zero).\\
\hline
  MINOR\_CEILING & round values exactly on the boundary by increasing the rational value (towards $\infty$).\\
\hline
  MINOR\_FLOOR   & round values exactly on the boundary by decreasing the rational value (towards $-\infty$).\\
\hline
  MINOR\_EVEN    & round values exactly on the boundary by using the choice resulting in the rounded last digit to be even.\\
\hline
  MINOR\_ODD     & round values exactly on the boundary by using the choice resulting in the rounded last digit to be odd.\\
\hline
\end{tabular}

\pagebreak

These combine to rounding modes:

\begin{tabular}{|l|p{100mm}|}
\hline
{\bfseries rounding mode}&{\bfseries description}\\
\hline
  ROUND\_UP           & always round in such a way that the absolute value does not decrease.\\
\hline
  ROUND\_DOWN         & always round in such a way that the absolute value does not increase.\\
\hline
  ROUND\_CEILING      & always round in such a way that the rational value does not decrease.\\
\hline
  ROUND\_FLOOR        & always round in such a way that the rational value does not increase.\\
\hline
  ROUND\_HALF\_UP      & round to the nearest available value, prefer increasing the absolute value if the last digit is 5\\
\hline
  ROUND\_HALF\_DOWN    & round to the nearest available value, prefer decreasing the absolute value if the last digit is 5\\
\hline
  ROUND\_HALF\_CEILING & round to the nearest available value, prefer increasing the rational value if the last digit is 5\\
\hline
  ROUND\_HALF\_FLOOR   & round to the nearest available value, prefer decreasing the rational value if the last digit is 5\\
\hline
  ROUND\_HALF\_EVEN    & round to the nearest available value, prefer the resulting last digit to be even if the last digit prior to rounding is 5\\
\hline
  ROUND\_HALF\_ODD    & round to the nearest available value, prefer the resulting last digit to be odd if the last digit prior to rounding is 5\\
\hline
  ROUND\_GEOMETRIC\_UP      & round to the available value using the geometric mean as boundary, prefer increasing the absolute value if the unrounded value is exactly on the boundary\\
\hline
  ROUND\_GEOMETRIC\_DOWN    & round to the available value, using the geometric mean as boundary, prefer decreasing the absolute value if the unrounded value is exactly on the boundary\\
\hline
  \dots & \ldots\\
\hline
  ROUND\_HARMONIC\_UP      & round to the available value using the harmonic mean as boundary, prefer increasing the absolute value if the unrounded value is exactly on the boundary\\
\hline
  ROUND\_HARMONIC\_DOWN    & round to the available value, using the harmonic mean as boundary, prefer decreasing the absolute value if the unrounded value is exactly on the boundary\\
\hline
  \dots & \ldots\\
\hline
  ROUND\_QUADRATIC\_UP      & round to the available value using the quadratic mean as boundary, prefer increasing the absolute value if the unrounded value is exactly on the boundary\\
\hline
  ROUND\_QUADRATIC\_DOWN    & round to the available value, using the quadratic mean as boundary, prefer decreasing the absolute value if the unrounded value is exactly on the boundary\\
\hline
  \dots & \ldots\\
\hline
  ROUND\_CUBIC\_UP      & round to the available value using the cubic mean as boundary, prefer increasing the absolute value if the unrounded value is exactly on the boundary\\
\hline
  ROUND\_CUBIC\_DOWN    & round to the available value, using the cubic mean as boundary, prefer decreasing the absolute value if the unrounded value is exactly on the boundary\\
\hline
  \dots & \ldots\\
\hline
  ROUND\_CUBIC\_ODD    & round to the available value using the cubic mean as boundary, prefer the resulting last digit to be odd if the unrounded value is exactly on the boundary\\
\hline
  ROUND\_UNNECESSARY  & raise an exception if value changing rounding would be needed\\
\hline
\end{tabular}


In addition to these commonly available rounding operations {\slshape long-decimal\/} provides rounding to remainder sets.  This is motivated by the practice in some
currencies to prefer using certain multiples of the smallest unit.  For example in CHF you commonly use two digits after the decimal point, but the last digit is
required to be 0 or 5.  This is achieved as a special case of a more general concept {\slshape round\_to\_allowed\_remainders\/}.  A modulus $M \ge 2$ and a set
$R\subset\mathbb{N}_0$ are given and a number $x$ is rounded to $\ld(n, d)$ such that

$$\bigvee_{r\in R} n \equiv r \mod M$$

Typically $M$ is ten or a power of ten, but this is not required.  Neither is it required that zero is a member of $R$.  In that case an additional parameter is needed
in order to define to which direction a potential last digit of zero would be rounded.  In the CHF case we would have $M=10$ and $R=\{0, 5\}$.

This kind of rounding can be applied to integers as well as to
{\slshape LongDecimal\/} and {\slshape LongDecimalQuot\/}, in which case the integral numerator
$n$ of $\ld(n, d)=\frac{n}{10^d}$ must fullfill the additional
constraint.  In order to avoid complications, this is not supported in
conjunction with minor rounding modes MINOR\_EVEN and MINOR\_ODD,
because all matching rounded values might be even or odd and we cannot
rely on the alternation of even and odd values.
It is possible to exclude $0$ from the set $R$. In this case a
ZERO-rounding-mode needs to be provided that tells in which way a
zero, should it occur, shoud be rounded.

\pagebreak

\section{Division}

Regular division using $/$ yields an instance of {\slshape LongDecimalQuot\/}.  The approximate number of significant digits is estimated by using the partial derivatives of $f(x, y) = \frac{x}{y}$.
Assume $x=\ld(m, s)$ and $y=\ld(n, t)$.  So we get for $\ld(m, s) / \ld(n, t)$ an result

$$\frac{x}{y} = \ldq(10^{t-s}\frac{m}{n}, r)$$

with

$$r \approx -\log_{10}\left(10^{-s} \frac{1}{|y|} + 10^{-t} \frac{|x|}{y^2}\right)
    =  2 \log_{10}(y) + s + t -\log_{10}\left(|m| + |n|) \right)$$

In order to avoid expensive logarithmic operation for such basic operations as division this is approximated by

$$r = \max( 0, 2 v + s + t - \max( u + s, v + t) - 3)$$
with $u = \lfloor\log |m| \rfloor - s + 1$ and $v = \lfloor\log |n| \rfloor - t + 1$ for $m$ and $n$ not zero
and $u=-s$ for $m = 0$ and $v=-t$ for $n=0$.  Using $\max(a, b)$ instead of $log_{10}(10^a + 10^b)$ is a good approximation, when a and b are far apart.

It is recommended to use explicit rounding after having performed
division rather than to rely on this somewhat arbitrary estimation on
the number of significant digits.  It is possible to retain
intermediate results with {\slshape LongDecimalQuot\/}, because the
full arithmetic is available for this type as well, but using rational
numbers internally it will blow up numerator and denominator to an
extent that diminishes performance by quite a margin in longer
calculations.

\pagebreak

\section{Roots}

Square root and cube root of {\slshape LongDecimal\/} can be calculated quite
efficiently using an algorithm that is somewhat similar to the
algorithm for long integer division.  This has been preferred over the
more commonly known Newton algorithm.  Since square and cube roots are
usually irrational, it is mandatory to provide rounding information
concerning the number of desired digits and the rounding mode.

Square roots and cube roots can also calculated of integers, in which
case a variant is available that also calculates a remainder $r$ in
addition to the approximated square root $s$, such
that
$a = s^2+r$.

\section{$\pi$}

The number $\pi$ can be calculated to a given number of digits, which
works sufficiently fast for a few thousend digits.

Please use dedicated programs if you seriously want to calculate $\pi$
to millions of digits, Ruby is not fast enough for this kind of number
crunching to compete with the best C-programs.

\pagebreak

\section{Transcendental Functions}

It is the goal of this library to support the most common
transcendental functions in the long run.  Currently $\exp$ and $\log$
are supported.  These are calculated using the Taylor series, but the
calculation has been improved.  Most important it is to improve the
convergence, which is the case with a series of the form

$$\sum_{n=0}^\infty \frac{x^n}{n!}$$

for $x< 0.5$.

For the exponential function this can always be achieved by using the following transformations: $x = 2^n x_0$ with $x_0 < 0.5$ implies

$$\exp(x) = \exp(x_0)^{2^n}$$

which can be easily calculated using successive squaring operations.

For the logarithm we first use $x = e^n x_0$ with $n \in \mathbb{N}_0$ and $x_0 < e$ with

$$\log(x) = n + \log(x_0).$$

From here we make use of the efficient square root calculation facility and use

$$\bigwedge_{m=0}^\infty x_{m+1} = \sqrt{x_m}$$

and

$$\log(x) = n + 2^m \log x_m$$

Using this $x_m$ can be brought sufficiently close to $1$ to make the Taylor series of $\log$ converge sufficiently fast to be useful for the calculation:

$$\log(x_m) = \sum_{k=1}^\infty \frac{(x_m-1)^k(-1)^{k+1}}{k}$$

Another minor optimization uses some integer $j$ and adds

$$ \bigwedge_{l=0}^{j-1} s_k = \sum_{l=0}^n \frac{x^{lj}}{(lj+k)!}$$

from which we finally multiply the partial sums with appropriate low powers of $x$.  This saves on the number of multiplications.
Similar patterns will be applied to the calculation of other transcendental functions.

\pagebreak

\section{Powers}

Calculation of powers with any positive {\slshape LongDecimal\/} as base and any {\slshape LongDecimal\/} as exponent is quite challenging to do.

It is quite trivial that $x^y = \exp(y \log x)$, but the challenges
are to achieve the result in acceptable time and with the required
precision.  The builtin class {\slshape BigDecimal} does not meet
these goals, because it becomes incredibly slow for certain
combinations of $x$ and $y$.  The power function of {\slshape LongDecimal\/} has
been optimized to handle a broad range of cases with different
approaches to provide precision and speed. This should work fine for
reasonable practical use, but it will still be possible to construct
corner cases with bases very close to 1 and large exponents which fail
in an attempt to do an accurate calculation in their last digits.

\section{Means}

In the class {\slshape LongMath} there are methods for calculating the
following means: arithmetic\_mean, geometric\_mean, harmonic\_mean,
quadratic\_mean, cubic\_mean, arithmetic\_geometric\_mean,
harmonic\_geometric\_mean.  See Wikipedia for their definitions.

\section{Rounding with sum constraint}

Experimental support for rounding of several number simultanously in
such a way that their rounded sum is the sum of the rounded numbers is
included with the methods {\slshape LongMath.round\_sum\_hm} which uses
the Haare-Niemeyer-approach and {\slshape LongMath.round\_sum\_divisor}
which uses one of several divisor based approaches, like D'Hondt. The
approach is chosen by the rounding mode. These are not yet implemented
efficiently nor are they tested well, so use at your own risk (like
the whole library).

\section{Limitations}

Rounding with sum constraint is not yet production stable.

Powers with bases that are off 1 by $10^{-20}$ or less and exponents
in the order of magnitude of $10^20$ or more are not always calculated
precisely.

Some interesting transcendental functions are missing.

Many operations (like power, exp, log and other transcendental functions) fail to work with numbers whose magnitude cannot be
expressed as double.

Using transcendental functions that have a rounding mode and precision
as part of their parameter set has not been tested with the newer
rounding modes, \dots\_ODD, ROUND\_GEOMETRIC\_\dots,
ROUND\_HARMONIC\_\dots, ROUND\_QUADRATIC\_\dots and
ROUND\_CUBIC\_\dots.
These combinations will be tested and improved in future versions.

\section{Tests}

Unit tests have been added to test much of the implemented
functionality.

More unit tests are desirable.

\end{document}
