% LaTeX
% $Header: /var/cvs/long-decimal/long-decimal/tex/long-decimal.tex,v 1.3 2011/02/09 22:51:52 bk1 Exp $
% $Name:  $
%
\documentclass[10pt,a4paper]{article}
\usepackage{isodate}
\usepackage[T1]{fontenc}
\usepackage[latin1]{inputenc}
\usepackage{graphics}
\usepackage{amsmath,amssymb}

\title{The Ruby library {\slshape long-decimal}}
\author{Karl Brodowsky\\
IT Sky Consulting GmbH\\
Switzerland}

\date{2011-02-06}

%\makeatletter
%\makeatother

\setlength{\topmargin}{0mm}
%\setlength{\leftmargin}{10mm}
\setlength{\oddsidemargin}{-11mm}
\setlength{\evensidemargin}{-11mm}
\setlength{\parindent}{0mm}

%\newfont{\ocrb}{ocrb10}

\def\ld{\mathrm l}
\def\ldq{\mathrm q}

\begin{document}
\sffamily
\maketitle

\begin{abstract}

The goal of the Ruby-library {\slshape long-decimal}\/ is to provide a new
numerical type {\slshape LongDecimal}\/ and arithmetic operations to deal
with this type.

\end{abstract}

\section{Introduction}

Ruby supports non-integral numbers with the built-in types {\slshape
  Float\/} and {\slshape Rational\/} and {\slshape BigDecimal\/}.
While these are very useful for many purposes, the development of
finance application requires the availability of a type like
LongDecimal, that allows explicit control of the rounding and uses a
decimal representation for the fractional part.  An instance of
{\slshape LongDecimal\/} can be represented as a pair of integral
numbers $(n, d)$, where the value of the number is $\frac{n}{10^d}$.
The ring-operations $+$, $-$ and $*$ can be trivially defined in a way
that does not loose any information (and thus does not require any
rounding).

Division requires some additional thought, because the exact quotient
of two {\slshape LongDecimal\/}s can very well be expressed as
Rational, but not always as {\slshape LongDecimal\/}.  A supplementary
numeric type {\slshape LongDecimalQuot\/} has been introduced to
support storing pairs $(r, d)$, where $r$ is a rational number and $d$
is an estimation about the significant digits after the decimal point.
The represented value is the rational number $r$.

Calculation of roots and trancendental functions require additional
information as input parameters to control the number of digits after
the decimal point and the rounding rules for achieving the result.

In the remainder of this document we will write $\ld(n, d)$ for the
{\slshape LongDecimal\/}-number represented by the pair $(n, d)$ in
the manner described above and $\ldq(r, d)$ for the {\slshape
  LongDecimalQuot\/}-number represented by teh pair $(r, n)$.  It
needs to be observed that the part $d$ in both cases carries
information about the precision of the number in terms of significant
digits after the decimal point.

\pagebreak

\section{Ring Operations}

The ring operations $+$, $-$ and $*$ and obvious derived operations
like negation are defined like this:

$$\bigwedge_{m,n \in \mathbb{N}_0} \bigwedge_{d, e \in \mathbb{Z}}
\ld(m,d) + \ld(n,e) = \ld(m \frac{\max(d, e)}{d} + n \frac{\max(d, e)}{e}, \max(d, e))$$
$$\bigwedge_{m,n \in \mathbb{N}_0} \bigwedge_{d, e \in \mathbb{Z}}
\ld(m,d) - \ld(n,e) = \ld(m \frac{\max(d, e)}{d} - n \frac{\max(d, e)}{e}, \max(d, e))$$
$$\bigwedge_{m,n \in \mathbb{N}_0} \bigwedge_{d, e \in \mathbb{Z}}
\ld(m,d) \cdot \ld(n,e) = \ld(m n, d+e)$$

\pagebreak

\section{Rounding Operations}

The library supports two rounding operations that can be applied both
to {\slshape LongDecimal\/} and {\slshape LongDecimalQuot\/}.  In the
latter case it implies a conversion to {\slshape LongDecimal\/}.  The
normal rounding function {\slshape round\_to\_scale\/} changes the
precision to the given value, while keeping the value expressed by the
number approximately the same.

The optional second parameter describes how rounding will be done, if the
process looses information.  It defaults to {\slshape
ROUND\_UNNECESSARY\/}, in which case an exception is raised whenever
a value changing rounding process would be required.

\begin{tabular}{|l|l|}
\hline
{\bfseries rounding mode}&{\bfseries description}\\
\hline
  ROUND\_UP           & always round in such a way that the absolute value does not decrease.\\
\hline
  ROUND\_DOWN         & always round in such a way that the absolute value does not increase.\\
\hline
  ROUND\_CEILING      & always round in such a way that the rational value does not decrease.\\
\hline
  ROUND\_FLOOR        & always round in such a way that the rational value does not increase.\\
\hline
  ROUND\_HALF\_UP      & round to the nearest available value, prefer increasing the absolute value if the last digit is 5\\
\hline
  ROUND\_HALF\_DOWN    & round to the nearest available value, prefer decreasing the absolute value if the last digit is 5\\
\hline
  ROUND\_HALF\_CEILING & round to the nearest available value, prefer increasing the rational value if the last digit is 5\\
\hline
  ROUND\_HALF\_FLOOR   & round to the nearest available value, prefer decreasing the rational value if the last digit is 5\\
\hline
  ROUND\_HALF\_EVEN    & round to the nearest available value, prefer the resulting last digit to be even if the last digit prior to rounding is 5\\
\hline
  ROUND\_UNNECESSARY  & raise an exception if value changing rounding would be needed\\
\hline
\end{tabular}

In addition to these commonly available rounding operations {\slshape long-decimal\/} provides rounding to remainder sets.  This is motivated by the practice in some
currencies to prefer using certain multiples of the smallest unit.  For example in CHF you commonly use two digits after the decimal point, but the last digit is
required to be 0 or 5.  This is achieved as a special case of a more general concept {\slshape round\_to\_allowed\_remainders\/}.  A modulus $M \ge 2$ and a set
$R\subset\mathbb{N}_0$ are given and a number $x$ is rounded to $\ld(n, d)$ such that

$$\bigvee_{r\in R} n \equiv r (\mod M)$$

Typically $M$ is ten or a power of ten, but this is not required.  Neither is it required that zero is a member of $R$.  In that case an additional parameter is needed
in order to define to which direction a potential last digit of zero would be rounded.  In the CHF case we would have $M=10$ and $R=\{0, 5\}$.

\pagebreak

\section{Division}

Regular division using $/$ yield an instance of {\slshape LongDecimalQuot\/}.  The approximate number of significant digits is estimated by using the partial derivatives of $f(x, y) = \frac{x}{y}$.
Assume $x=\ld(m, s)$ and $y=\ld(n, t)$.  So we get for $\ld(m, s) / \ld(n, t)$ an result 

$$\frac{x}{y} = \ldq(10^{t-s}\frac{m}{n}, r)$$

with 

$$r \approx -\log_{10}\left(10^{-s} \frac{1}{|y|} + 10^{-t} \frac{|x|}{y^2}\right) 
    =  2 \log_{10}(y) + s + t -\log_{10}\left(|m| + |n|) \right)$$

In order to avoid expensive logarithmic operation for such basic operations as division this is approximated by

$$r = \max( 0, 2 v + s + t - \max( u + s, v + t) - 3)$$
with $u = \lfloor\log |m| \rfloor - s + 1$ and $v = \lfloor\log |n| \rfloor - t + 1$ for $m$ and $n$ not zero 
and $u=-s$ for $m = 0$ and $v=-t$ for $n=0$.  Using $\max(a, b)$ instead of $log_{10}(10^a + 10^b)$ is a good approximation, when a and b are far apart.

It is recommended to use explicit rounding after having performed
division rather than to rely on this somewhat arbitrary estimation on
the number of significant digits.  It is possible to retain
intermediate results with {\slshape LongDecimalQuot\/}, because the
full arithmetic is available for this type as well, but using rational
numbers internally it will blow up numerator and denominator to an
extent that diminishes performance by quite a margin in longer
calculations.

\pagebreak

\section{Roots}

Square root and cube root of LongDecimal can be calculated quite
efficiently using an algorithm that is somewhat similar to the
algorithm for long integer division.  This has been preferred over the
more commonly known Newton algorithm.  Since square and cube roots are
usually irrational, it is mandatory to provide rounding information
concerning the number of desired digits and the rounding mode.

\section{$\pi$}

The number $\pi$ can be calculated to a given number of digits, which
works sufficiently fast for a few thousend digits.

Please use dedicated programs if you seriously want to calculate $\pi$
to millions of digits, Ruby is not fast enough for this kind of number
crunching to compete with the best C-programs.

\pagebreak

\section{Transcendental Functions}

It is the goal of this library to support the most common
transcendental functions in the long run.  Currently $\exp$ and $\log$
are supported.  These are calculated using the Taylor series, but the
calculation has been improved.  Most important it is to improve the
convergence, which is the case with a series of the form

$$\sum_{n=0}^\infty \frac{x^n}{n!}$$

for $x< 0.5$.

For the exponential function this can always be achieved by using the following transformations: $x = 2^n x_0$ with $x_0 < 0.5$ implies

$$\exp(x) = \exp(x_0)^{2^n}$$

which can be easily calculated using successive squaring operations.

For the logarithm we first use $x = e^n x_0$ with $n \in \mathbb{N}_0$ and $x_0 < e$ with

$$\log(x) = n + \log(x_0).$$

From here we make use of the efficient square root calculation facility and use 

$$\bigwedge_{m=0}^\infty x_{m+1} = \sqrt{x_m}$$

and

$$\log(x) = n + 2^m \log x_m$$

Using this $x_m$ can be brought sufficiently close to $1$ to make the Taylor series of $\log$ converge sufficiently fast to be useful for the calculation:

$$\log(x_m) = \sum_{k=1}^\infty \frac{(x_m-1)^k(-1)^{k+1}}{k}$$

Another minor optimization uses some integer $j$ and adds

$$ \bigwedge_{l=0}^{j-1} s_k = \sum_{l=0}^n \frac{x^{lj}}{(lj+k)!}$$

from which we finally multiply the partial sums with appropriate low powers of $x$.  This saves on the number of multiplications.
Similar patterns will be applied to the calculation of other transcendental functions.

\pagebreak

\section{Powers}

Calculation of powers with any positive {\slshape LongDecimal} as base and any {\slshape LongDecimal} as exponent is quite challenging to do.

It is quite trivial that $x^y = \exp(y \log x)$, but the challenges
are to achieve the result in acceptable time and with the required
precision.  The builtin class {\slshape BigDecimal} does not meet
these goals, because it becomes incredibly slow for certain
combinations of $x$ and $y$.  

\section{Limitations}

\section{Tests}

\end{document}
